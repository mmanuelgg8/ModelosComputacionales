\documentclass[12pt]{article}
\title{Práctica 5: Perceptrón}
\author{Manuel González González}
\begin{document}
\maketitle
\section*{Ejercicio 3}
Si tiene una baja tasa de aprendizaje tarda más en encontrar la solución, ya que cada modificación realizada es muy pequeña, y si se encuentra lejos de esta necesitará más repeticiones del bucle.
\section*{Ejercicio 4}
Llega al máximo de épocas sin haber encontrado la solución óptima. Esto se debe a que el problema no es linealmente separable, y el perceptrón no es capaz de separar la región en dos zonas que cumplan el requisito.
\section*{Ejercicio 5}
Con el punto (-1,-1,1) no sería capaz de encontrar solución, pues no se trataría de un problema linealmente separable.
\end{document}